\documentclass{book}

% Packages for formatting and indexing
\usepackage{amsmath,amssymb,amsfonts}
\usepackage{hyperref}
\usepackage{makeidx}
\usepackage[shortlabels]{enumitem}
\makeindex

% Custom environments
\newtheorem{definition}{Definition}[section]
\newtheorem{lemma}[definition]{Lemma}
\newtheorem{corollary}[definition]{Corollary}
\newtheorem{theorem}[definition]{Theorem}
\newtheorem{example}[definition]{Example}
\newtheorem{note}[definition]{Note}
\newtheorem{proposition}[definition]{Proposition}

\newcommand{\R}{\mathbb{R}}
\newcommand{\Z}{\mathbb{Z}}
\newcommand{\Q}{\mathbb{Q}}
\newcommand{\lr}[1]{\langle #1 \rangle}

% Custom commands for formatting and indexing
\newcommand{\myref}[1]{\nameref{#1} (\textbf{\ref{#1}}) \index{\nameref{#1}}}

% Creating custom environments to automatically do labels and index
\newenvironment{mydefn}[1]{
  \def\myarg{#1} 
  \begin{definition}[#1]\label{defn:#1}
}{
  \index{\nameref{defn:\myarg}}
  \end{definition}
}

\newenvironment{mylemma}[1]{
  \def\myarg{#1} 
  \begin{lemma}[#1]\label{lem:#1}
}{
  \index{\nameref{lem:\myarg}}
  \end{lemma}
}

\newenvironment{mycorollary}[1]{
  \def\myarg{#1} 
  \begin{corollary}[#1]\label{cor:#1}
}{
  \index{\nameref{cor:\myarg}}
  \end{corollary}
}

\newenvironment{mytheorem}[1]{
  \def\myarg{#1} 
  \begin{theorem}[#1]\label{thm:#1}
}{
  \index{\nameref{thm:\myarg}}
  \end{theorem}
}

\newenvironment{myexample}[1]{
  \def\myarg{#1} 
  \begin{example}[#1]\label{ex:#1}
}{
  \index{\nameref{ex:\myarg}}
  \end{example}
}

\newenvironment{mynote}[1]{
  \def\myarg{#1} 
  \begin{note}[#1]\label{note:#1}
}{
  \index{\nameref{note:\myarg}}
  \end{note}
}

\newenvironment{myprop}[1]{
  \def\myarg{#1} 
  \begin{proposition}[#1]\label{prop:#1}
}{
  \index{\nameref{prop:\myarg}}
  \end{proposition}
}

\begin{document}

\frontmatter

% Title page
\title{Math 791 Summation}
\author{Will Thomas}
\date{\today}
\maketitle

% Table of contents
\tableofcontents

\mainmatter

% Introduction
\chapter{Introduction}

This is the introduction to my math textbook.

% Chapter 1
\chapter{Groups}

This chapter will focus primarily on Groups and group related theories.

\section{Basic Definitions}

\begin{mydefn}{Group}
  A set $G$ with a binary operation $* : G \times G \rightarrow G$ such that
  \begin{enumerate}[1.]
    \item $\exists e \in G$ s.t. $\forall g \in G,\ g * e = e * g = g$ (identity element)

    \item $\forall g_1\ g_2\ g_3 \in G,\ g_1 * (g_2 * g_3) = (g_1 * g_2) * g_3$ (associativity)

    \item $\forall g \in G, \exists g^{-1} \in G$ s.t. $g * g^{-1} = e = g^{-1} * g$ (inverses)
  \end{enumerate}
\end{mydefn}

A good example of a \myref{defn:Group} is $GL_n(\R)$, which is the general linear group of matrices that are all invertible for a given field $\R$.

\begin{mydefn}{Trivial}
  A \emph{trivial} group $G$ is a group with only 1 element.
  Namely, this element must be the identity element.
\end{mydefn}

\begin{mydefn}{Abelian}
  A \myref{defn:Group} $G$ is said to be "Abelian" if
  $$\forall g_1\ g_2 \in G,\ g_1 * g_2 = g_2 * g_1$$
  Succinctly, all elements in that group commute.
\end{mydefn}

\section{Symmetric Groups}

Additionally, we can consider the notion of Symmetric Groups:
\begin{mydefn}{Permutation Group}
  Let $X$ be a set with $n$ elements, we can define
  $$S_n := \{ \sigma : X \rightarrow X \mid \sigma \text{ is 1-1 and onto} \}$$
  Which is the "Permutation Group"
\end{mydefn}

The \myref{defn:Permutation Group} is a group under function composition (so the operation $*$ is composition) and additionally, $|S_n| = n\!$

\begin{mydefn}{Dihedral Groups}
  We can define $D_n$ as the group of symmetries of a regular $n-$gon (a $n$ sided polygon).

  This consists of rotations about center and reflections about lines of symmetry.
  All such rotations must land back on itself, aka just the vertices move.
\end{mydefn}

We also know that the following major properties hold for groups:
\begin{enumerate}[(i.)]
  \item Uniqueness of identity element
  \item Uniqueness of inverses
  \item $\forall g \in G,\ (g^{-1})^{-1} = g$
\end{enumerate}

\begin{mydefn}{Subgroup}
  A subset $H \subseteq G$ is a subgroup if:
  \begin{enumerate}[(i.)]
    \item $H$ is closed under the binary operations of $G$
    \item $\forall h \in H, h^{-1} \in H$
  \end{enumerate}
\end{mydefn}

\begin{mydefn}{The Generated Subgroup}
  Let $X \subseteq G$ be a subset. Then the "subgroup of $G$ generated by $X$" (denoted $\langle X \rangle$), is the set of all finite expressions of the form $x_1^{\epsilon_1} \cdots x_{r}^{\epsilon_r}$, where all $x_i \in X$ and $\epsilon = -1, 0, 1$.
\end{mydefn}

\begin{mytheorem}{Subgroup Intersection Theorem}
  $\langle X \rangle = $ the intersectiin of all subgroups of $G$ containing $X$.
\end{mytheorem}

One thing to note is that if $X := \{ a \}$ (it is a single element), then we call $\lr{X} = \lr{a} = $ \emph{the cyclic subgroup of $G$ generated by $a$}.

\begin{mydefn}{Cosets}
  If we have a \myref{defn:Subgroup} $H \subseteq G$, we can define:

  \emph{Left Coset} $gH := \{ gh \mid h \in H \}$ and the \emph{Right Coset} $Hg := \{ hg \mid h \in H \}$.
\end{mydefn}

\begin{mytheorem}{Coset Partition Theorem}
  All distinct left (respectively right) cosets of $H$ in $G$ partition $G$.
\end{mytheorem}

An additional critical properties of groups is the "Cancelation Property",
which implies that for any group $G$, $\forall g\ x\ y \in G, gx = gy \implies x = y$.

This property can be used to see that there is a 1-1 function from a subgroup $H$ to $gH$ for any $g \in G$. Thus $|H| = |gH|$.

Due to \myref{thm:Coset Partition Theorem} we can now show major result

\begin{mytheorem}{LaGrange's Theorem}
  Let $G$ be a finite group and $H \subseteq G$ a subgroup.

  Then:
  \begin{align*}
    |G| & = |H| \cdot \text{(number of distinct left cosets of $H$)}  \\
        & = |H| \cdot \text{(number of distinct right cosets of $H$)}
  \end{align*}
\end{mytheorem}

An immediately obvious corollary of this is that the number of distinct left cosets equals the number of distinct right cosets. This allows us to define the following notation $[G : H]$ \emph{the index of $H$ in $G$}, which represents the number of distinct cosets of $H$ in $G$.

\begin{mydefn}{Normal Subgroup}
  A subgroup $N$ is called \emph{normal} if
  $$\forall g \in G, n \in N, \exists n' \in N, gn = n'g$$
  Essentially allowing a form of commutativity, at the expense of changing $n$ to $n'$.
\end{mydefn}

\begin{mydefn}{Simple}
  A group $G$ is \emph{simple} if the only \myref{defn:Normal Subgroup}'s of $G$ are $G$ and the \myref{defn:Trivial Group}
\end{mydefn}

\begin{mydefn}{Quotient Group}
  The quotient group (also called factor group) of $G$ by $N$ is the set of left cosets of $N$ under $G$.

  This will be denoted as $G/N$ or $G \mod N$.

  This forms a group under the binary operation "coset multiplication" where $g_1Ng_2N = g_1g_2N$
\end{mydefn}

\section{Group Homomorphisms}

\begin{mydefn}{Group Homomorphism}
  Given groups $G_1, G_2$, a function $\phi : G_1 \rightarrow G_2$ is a \emph{group homomorphism} if
  $$\forall a\ b \in G_1,\ \phi(a *_1 b) = \phi(a) *_2 \phi(b)$$
  Where $*_1$ is the binary operation of $G_1$ and $*_2$ is the binary operation in $G_2$
\end{mydefn}

Some noteworthy properties of a group homomorphism that can be proven fairly trivially are:
\begin{enumerate}[(i.)]
  \item $\phi(e_1) = e_2$ (where $e_i$ represents the identity element from a respective group)
  \item $\forall g \in G,\ \phi(g^{-1}) = \phi(g)^{-1}$ (the inverse elements are preserved by a homomorphism)
\end{enumerate}

If we take the classic definition of \emph{Kernel} where $\ker(\phi) := \{ g \in G_1 \mid \phi(g) = e_2 \}$, we uncover some nice properties.

\begin{myprop}{Group Hom. Properties}
  Let $\phi : G_1 \rightarrow G_2$ be a group hom. (homomorphism), with kernel $K$.
  Then:
  \begin{enumerate}[(i)]
    \item $K$ is a \myref{defn:Normal Subgroup} of $G_1$
    \item If $H$ is a subgroup of $G_1$, then $\phi(H)$ is a subgroup of $G_2$. Essentially, the structure preserving properties of a Group Hom. are so strong, they preserve subgroup orderings!
  \end{enumerate}
\end{myprop}

Furthermore, we can find properties of Normal subgroups.
\begin{myprop}{Normal Subgroup Properties}
  \begin{enumerate}[(i.)]
    \item If $H$ is a normal subgroup of $G_1$, then $\phi$ being surjective $\implies$ $\phi(H)$ is normal as well
    \item Any normal subgroup $N$ of a group $G$ is the kernel of a group homomorphism. This homomorphism can be discovered as $\phi : G \rightarrow G/N$ defined by $\phi(g) = gN$.
  \end{enumerate}
\end{myprop}

\begin{mytheorem}{Surjective Group Hom. Theorem}
  Let $\phi : G_1 \rightarrow G_2$ be a surjective group homomorphism with kernel $K$. Then:
  There is a 1-1 correspondence between the subgroups of $G_1$ containing $K$ and the subgroups of $G_2$ given by $H \rightarrow \phi(H)$ for $H \subseteq G_1$ containing $K$ and $L \rightarrow \phi^{-1}(L)$ for $L \subseteq G_2$.
  Under this correspondence, $\phi(H)$ is normal in $G_2$ if $H$ is normal in $G_1$ and $\phi^{-1}(L)$ is normal in $G_1$, if $L$ is normal in $G_2$.

  Restated more simply: In an onto group hom. normal subgroups are preserved by the mapping.
\end{mytheorem}

\begin{mycorollary}{Normal Subgroup Theorem}
  Let $G$ be a group and $N$ a normal subgroup. Then:
  There is a 1-1 correspondence between the subgroups of $G$ containing $N$ and the subgroups of $G/N$.
  Under this correspondence, the normal subgroups of $G$ containing $N$ correspond to the normal subgroups of $G/N$.

  This is fairly straightforward when we take \myref{thm:Surjective Group Hom. Theorem} where $\phi : G \rightarrow G/N$ as $\phi(g) = gN$
\end{mycorollary}

\section{Isomorphisms}
First we will look into the primary Isomorphism Theorems (as they specifically apply to Groups)

\begin{mydefn}{Isomorphism}
  A \myref{defn:Group Homomorphism} $\phi$ that is also 1-1, and onto, is an \emph{Isomorphism}
\end{mydefn}

\begin{mytheorem}{First Isomorphism Theorem}
  Given $\phi : G_1 \rightarrow G_2$ a surjective (onto) group hom. with kernel $K$.
  Then:
  $G_1/K \cong G_2$.
\end{mytheorem}

\begin{mytheorem}{Second Isomorphism Theorem}
  Let $K \subseteq N \subseteq G$ be groups such that $K$ and $N$ are normal in $G$.
  Then:
  $N/K$ is a normal subgroup of $G/K$ and $(G/K)/(N/K) \cong G/N$
\end{mytheorem}

\begin{mytheorem}{Third Isomorphism Theorem}
  Given $H, K \subseteq G$ (all groups), and $K$ is \myref{defn:Normal Subgroup} of $G$.
  Then:
  $$HK/K \cong H/(H \cap K)$$
\end{mytheorem}

\begin{mytheorem}{Symmetric Group Cycle Theorem}
  Let $\sigma \in S_n$ (the \myref{defn:Permutation Group}).
  Then:
  \begin{enumerate}[(i)]
    \item $\sigma$ can be written uniquely (up to order) as a product of disjoint cycles
    \item $\sigma$ can be written as a product of (not necessarily disjoint) $2-$cycles.
    \item This rewriting as a product of $2-$cycles is guaranteed to preserve the degree of the order (even or odd) no matter the way it is rewritten.
  \end{enumerate}
\end{mytheorem}

\begin{mydefn}{Alternating Group}
  The \emph{Alternating Group} is the set of all "even" order permutations on a set with $n$ elements. This is typically represented as $A_n$
\end{mydefn}

\begin{mytheorem}{Alternating Simple Group Theorem}
  $A_n$ is a simple group for $n \geq 5$.

  This can be intuitively stated as "there are no proper normal subgroups of $A_n$ for $n \geq 5$.
\end{mytheorem}

\section{Group Actions}

\begin{mydefn}{Group Action}
  Given a set $X$ and a group $G$, we say \emph{$G$ acts on $X$} if:
  \begin{enumerate}[(i.)]
    \item There is a binary map $G \times X \rightarrow X$ with the below properties
    \item $e \cdot x = x$
    \item $\forall a,b \in G,\ (ab) \cdot x = a \cdot (b \cdot x)$
  \end{enumerate}
\end{mydefn}

\begin{myprop}{Group Action Homomorphism}
  A group $G$ acts on a set with $n$ elements.
  $\iff$ There exists a group homomorphism $\phi : G \rightarrow S_n$
\end{myprop}

\begin{mytheorem}{Prime Degree Subgroups}
  If $G$ is a finite group, $H \subseteq G$ a subgroup, and $[G : H] = p$, where $p$ is the smallest prime dividing the order of $G$.
  Then: $H$ is normal in $G$
\end{mytheorem}

\begin{mydefn}{Orbit}
  The \emph{Orbit of $x \in X$} is defined as $orb := \{ g \cdot x \mid g \in G \}$
  Given that $G$ acts on $X$.
\end{mydefn}

\begin{mydefn}{Stabilizer}
  The \emph{Stabilizer of $x$} is defined as $G_x := \{ g \in G \mid g \cdot x = x \}$
  Given $G$ acts on $X$.
\end{mydefn}

It is worth noting that all the distinct orbits under the a given action will partition $X$. This will follow immediately from the following proposition

\begin{myprop}{Orbit Correspondence Theorem}
  Given a group $G$ acting on a set $X$. If we fix $x \in X$, there is a 1-1, onto set map between $orb(x)$ and the set of distinct left cosets of $G_x$.

  Furthermore, this can be given by $g \cdot x \rightarrow gG_x$.
  Allowing us to see that whenever $|orb(x)|$ or $[G : G_x]$ is finite, then
  $|orb(x)| = [G : G_x]$
\end{myprop}

\begin{myprop}{Groups acting via Conjugation}
  In the case that $G$ acts on itself via conjugation ($g \cdot x := gxg^{-1}$)
  then $orb(x) = \{ gxg^{-1} \mid g \in G \}$ which is then called the \emph{Conjugacy Class of $G$} which we denote $c(x)$.
  Then $G_x := \{ g \in G \mid gx = xg \}$ which is called the \emph{Centralizer of $X$}, and we denote $C_G(x)$.

  Thus $|c(x)| = [G : C_G(x)]$ if either is finite.
\end{myprop}

\begin{mydefn}{Center of a Group}
  For a given group $G$, we can take $Z(G)$ to be the \emph{center of $G$}
  and it is defined as $Z(G) := \{ g \in G \mid gx = xg \}$.
\end{mydefn}

\begin{mytheorem}{Class Equation}
  Let $G$ be a finite group.
  Then:
  \begin{align*}
    |G| & = |Z(G)| + \sum_{i = 1}^{r} |c(x_i)|       \\
        & = |Z(G)| + \sum_{i = 1}^{r} [G : C_G(x_i)]
  \end{align*}
\end{mytheorem}

\begin{mytheorem}{Groups of Prime Power Orders}
  Let $G$ be a finite group where $|G| = p^n$ ($p$ is prime) and $n \geq 1$.
  Then:
  \begin{enumerate}[(i)]
    \item $Z(G) \neq \{ e \}$
    \item For each $1 \leq i < n$, $G$ has a subgroup of order $p^i$
  \end{enumerate}
\end{mytheorem}

\section{Sylow Theorems}

\begin{mytheorem}{First Sylow Theorem}
  Let $G$ be a finite group such that $|G| = p^nm$, where $p$ is prime and $p$ does not divide $m$.
  Then:
  $G$ has a Sylow $p-$subgroup. Which means there exista a subgroup $P \subseteq G$ such that $|P| = p^n$.
\end{mytheorem}

\begin{mycorollary}{Sylow Corollary}
  A couple neat properties fall out of this theorem.
  \begin{enumerate}[(i)]
    \item If $|G| = p^nm$, then for each $1 \leq i \leq n$, there exist subgroups $H_1 \subseteq \cdots \subseteq H_n$ such that $|H_i| = p^i$.
          Essentially allowing us to know that there is a descending chain of prime power order subgroups.
    \item If $|G| = pq^n$, ($p,q$ both prime) and $p < q$, then $G$ has a normal Sylow $q-$subgroup (which is the unique Sylow $q$-subgroup).
  \end{enumerate}
\end{mycorollary}

\begin{mytheorem}{Second Sylow Theorem}
  Let $G$ be a finite gorup such that $|G| = p^nm$, where $p$ is prime and $p$ does not divide $m$.
  Supposed $H \subseteq G$ is a subgroup of order $p^i$, with $1 \leq i \leq n$ and $P$ is a Sylow $p-$subgroup.
  Then:
  $\exists a \in G,\ H \subseteq aPa^{-1}$

  Essentially allowing us to conclude that any two Sylow $p$-subgroups are conjugate.
\end{mytheorem}

\begin{mytheorem}{Third Sylow Theorem}
  Let $G$ be a finite group such that $|G| = p^nm$, where $p$ is prime and $p$ does not divide $m$ and write $n_p$ for the number of Sylow $p$-subgroups.
  Then:
  $n_p$ divides $|G|$ and is congruent to $1 \mod p$.
\end{mytheorem}

Overall, the Sylow Theorems can be very helpful for showing that groups of an order like $p^nm$ will have a non-trivial normal subgroup.

\begin{mylemma}{Orbits for Prime Power Fields}
  Let $G$ be a group of order $p^t$ ($p$ prime) and assume $G$ acts on the finite set $X$.
  If $r$ denotes the number of orbits with just one element, then $|X| = r \mod p$
\end{mylemma}

\begin{mytheorem}{Simple Group of Order 60}
  Let $G$ be a simple group of order $60$.
  Then:
  $G \cong A_5$
\end{mytheorem}

\chapter{Rings}

\section{Basic Definitions}

\begin{mydefn}{Ring}
  A ring $R$ is a set $X$ with two binary operations $+$ and $\cdot$ such that the following properties hold:
  \begin{enumerate}[(i)]
    \item $(R, +)$ is an \myref{defn:Abelian} group
    \item Multiplication ($\cdot$) is associative
    \item $\forall a,b,c \in R,\ a \cdot (b + c) = a \cdot b + a \cdot c$ and $(b + c) \cdot a = b \cdot a + c \cdot a$
    \item $R$ has a multiplicative identity, denoted as $1$ satisfying \\
          $\forall a \in R,\ 1 \cdot a = a = a \cdot 1$
  \end{enumerate}
\end{mydefn}

\begin{mydefn}{Ideals}
  Given a ring $R$, a left ideal $I \subseteq R$ is a set satisfying:
  $$\forall i \in I, \forall r \in R, ir \in I$$
  Similarly, a right ideal $I \subseteq R$ is a set satisfying:
  $$\forall i \in I, \forall r \in R, ri \in I$$

  A natural further definition is a two-sided ideal (sometimes just referred to as an ideal), which is an $I \subseteq R$ that is both a left and a right ideal.
\end{mydefn}

\begin{mydefn}{Generated Ideals}
  Given a ring $R$ and a set $X \subseteq R$, the \emph{left ideal of $R$ generated by $X$} that we denote $\lr{X}_L$ is the interesection of all left ideals of $R$ containing $X$.
  This could also be characterized as all finite expressions $\forall r_i \in R, \forall x_i \in X,\ r_1x_1 + \cdot + r_nx_n$

  It is fairly straightforward to see what a corresponding \emph{right ideal of $R$} or \emph{two sided ideal of $R$} \emph{generated by $X$}.
\end{mydefn}

One could see the connections between \myref{defn:Normal Subgroup}'s and two-sided \myref{defn:Ideals}. For any two sided ideal, the abelian group $(R/I, +)$ has a ring structure when we look at coset multiplication. Very similarly to the way it operates on normal subgroups.

\section{Ring Homomorphisms}

\begin{mydefn}{Ring Homomorphism}
  A mapping $f : R \rightarrow S$ (where $R,S$ are rings) is a \emph{ring homomorphism} if the mapping preserves the structure within the ring.
\end{mydefn}

Similar results as the \myref{thm:First Isomorphism Theorem} through the \myref{thm:Third Isomorphism Theorem} can be generalized to the ring world.

\begin{mytheorem}{Fundamental Theorem of Arithmetic}
  Every positive integer $n$ can be written uniquely as a product $n = p_1^{e_1} \cdots p_r^{e_r}$ where each $p_i$ is prime and $e_i \geq 1$.

  The uniqueness of this statement means that if $n = q_1^{f_1} \cdots q_s^{f_s}$ and $n = p_1^{e_1} \cdots p_r^{e_r}$ then after re-indexing, $q_i = p_i$ and $e_i = f_i$ and $r = s$.
\end{mytheorem}

A corollary of this generalized to any field can be created

\begin{mycorollary}{General Field FTA}
  For a given field $F$, every monic polynomial $f(x) \in F[x]$ can be factored uniquely as a product $f(x) = p_1(x)^{e_1} \cdots p_r(x)^{e_r}$ where each $p_i(x)$ is a monic irreducible polynomial in $F$.

  This all hinges on the key fact that $F[x]$ will always have a division algorithm.
\end{mycorollary}

\section{Integral Domains}

\begin{mydefn}{Integral Domain}
  A commutative ring $R$ is an \emph{integral domain} (ID) if the product of non-zero elements is always non-zero.
\end{mydefn}


\begin{mydefn}{Unit}
  A \emph{unit} is an element within a ring that has a multiplicative inverse.
  That is, any $u \in R$ is a unit if $\exists u' \in R,\ uu' = 1 = u'u$
\end{mydefn}

\begin{mydefn}{Prime}
  When in an integral domain $R$, an element $p \in R$ is \emph{prime} if
  $p \mid ab \implies p \mid a \lor p \mid b$
\end{mydefn}

\begin{mydefn}{Irreducible}
  When in an integral domain $R$, an element $q \in R$ is \emph{irreducible} if
  $q = ab \implies a \lor b$ is a \myref{defn:Unit}
\end{mydefn}

\begin{myprop}{Integral Domain Properties}
  This lends to some useful properties in integral domains:
  \begin{enumerate}[(i)]
    \item Cancelation: $\forall a, b, c \in R$ (where $R$ is an \myref{defn:Integral Domain}), $a \neq 0 \land ab = ac \implies b = c$
  \end{enumerate}
  Additionally, the following are equivalent:
  \begin{enumerate}[(a)]
    \item Every non-zero, non-unit element of $R$ can be written as a product of \myref{defn:Prime} elements.
    \item Every non-zero, non-unit elements in $R$ can be written uniquely (up to order and unit multiples) as a product of irreducible elements.
  \end{enumerate}
\end{myprop}

\begin{mydefn}{Unique Factorization Domain}
  A \emph{Unique Factorization Domain} or (UFD) as a ring that satisfies the unique factorization properties laid out in \myref{prop:Integral Domain Properties}
\end{mydefn}

\begin{mydefn}{Principal Ideal Domain}
  A \emph{Principal Ideal Domain} or (PID) is any ring with a division algorithm.
\end{mydefn}

\begin{myprop}{Principal Ideal Properties}
  For any \myref{defn:Integral Domain} $R$
  \begin{enumerate}[(i)]
    \item $a \mid b \iff \lr{b} \subseteq \lr{a}$
    \item $\lr{a} = \lr{b} \iff b = au$ for some unit $u \in R$
    \item $q \in R$ is irreducible $\iff \lr{q}$ is maximal among principal ideals.
    \item $p \in R$ is prime $\iff ab \in \lr{p} \implies a \in \lr{p} \lor b \in \lr{p}$
  \end{enumerate}
\end{myprop}

A set of useful propositions over a PID are as following
\begin{myprop}{PID Propositions}
  For an ID $R$
  \begin{enumerate}
    \item $R$ satisfies the ascending chain condition on principal ideals
    \item Every non-empty collection of principal ideals has a maximal element
    \item Every non-zero, non-unit in $R$ is a product of finitely many irreducible elements.
  \end{enumerate}
  $$((i) \iff (ii)) \implies (iii)$$
  For the next two $R$ is a PID
  \begin{enumerate}[(a)]
    \item $R$ satisfies the ascending chain condition on principal ideals
    \item Every irreducible element is a prime element.
  \end{enumerate}
\end{myprop}

These can all be combined into the ultimate theorem
\begin{mytheorem}{PID UFD Theorem}
  Every PID is a UFD.
\end{mytheorem}

\section{Advanced Ring Theorems}

\begin{mydefn}{Quotient Field}
  A \emph{Quotient Field} $K$ can be constructed from any arbitrary integral domain $R$.
  If we take $R* = R \setminus { 0 }$ (removing the element $0$), then we can define an equivalence relation on $R \times R*$ by letting $(n, d) \sim (m,b) \iff nb = md$

  Using this, we can form the Quotient Field $K = (R \times R*, +, \cdot)$ where any two elements are equivalent via the above definition.

  This may also be called the field of fractions.
\end{mydefn}

\begin{myprop}{UFD Prime Element Proposition}
  For a UFD $R$, if $p \in R$ is prime, then $p$ is also prime in $R[x]$
\end{myprop}

\begin{mydefn}{Primitive}
  A polynomial $f(x)$ is primitive if the Greatest Common Divisor of all coefficients of the polynomial is $1$.

  This is more simply stated as "no prime number divides this element"
\end{mydefn}

\begin{mylemma}{Gauss's Primitive Polynomial Lemma}
  Let $R$ be a UFD.
  Then:
  The product of primitive polynomials is \myref{defn:Primitive}
\end{mylemma}

We can build some further propositions in the ultimate goals of proving that the UFD property for a ring can extend to a polynomial ring over that ring.

\begin{myprop}{Quotient Field Irreducible Element Proposition}
  Suppose $R$ is a UFD with a \myref{defn:Quotient Field} $K$ and $f(x) \in R[x]$ is primitive.
  Then:
  $f(x)$ is irreducible in $R[x] \iff $ it is irreducible in $K[x]$
\end{myprop}

\begin{myprop}{UFD Prime Element Proposition}
  Suppose $R$ is a UFD and $f(x) \in R[x]$ is primitive and irreducible.
  Then:
  $f(x)$ is a \myref{defn:Prime} element
\end{myprop}

\begin{mytheorem}{UFD Polynomial Ring Theorem}
  If $R$ is a UFD.
  Then:
  $R[x]$ is a UFD.
\end{mytheorem}

\begin{mydefn}{Eisenstein's Criterion}
  Given a polynomial $f(x) = a_nx^n + \cdots + a_1x + a_0$ with integer coefficients

  If $\exists p$ (prime) such that:
  \begin{enumerate}[(i)]
    \item $\forall 0 \leq i < n,\ p \mid a_i$
    \item $p \nmid a_n$
    \item $p^2 \nmid a_0$
  \end{enumerate}

  Then $f(x)$ is \myref{defn:Irreducible} over the rational numbers.
\end{mydefn}

\begin{myprop}{Commutative Ring Properties}
  We have some nice properties if we are a commutative ring
  \begin{enumerate}[(i)]
    \item An ideal $P \subseteq R$ is a prime ideal $\iff$ $R/P$ is an \myref{defn:Integral Domain}
    \item An ideal $M \subseteq R$ is a maximal ideal $\iff$ $R/M$ is a field
  \end{enumerate}
\end{myprop}

\begin{mydefn}{Noetherian}
  A commutative ring that satisfies any one of the following equivalent conditions is considered \emph{Noetherian}
  \begin{enumerate}[(i)]
    \item $R$ satisfies the ascending chain condition.
    \item $R$ satisfies the maximal condition (where any collection of ideals has a maximal element)
    \item Every ideal of $R$ is finitely generated.
  \end{enumerate}
\end{mydefn}

\begin{mytheorem}{Hilbert's Basis Theorem}
  Let $R$ be a Noetherian commutative ring.

  Then:
  $R[x]$ is \myref{defn:Noetherian}
\end{mytheorem}

\chapter{Fields}

\section{Basic Definitions}

\begin{mydefn}{Field}
  A \emph{Field} $F$ is a commutative ring where every non-zero element have a multiplicative inverse.

  Note: If $F$ is a field, $F$ is also an \myref{defn:Integral Domain}
\end{mydefn}

\begin{mydefn}{Degree}
  If $F \subseteq K$ are fields, $K$ can be regarded as a vector space over $F$.

  We refer to the dimension of this vector space $K$ over $F$ as \emph{the degree of $K$ over $F$}.

  We denote this $[K : F]$
\end{mydefn}

\begin{mydefn}{Algebraic}
  Let $F \subseteq K$ be a field, and $\alpha \in K$.
  Then:
  $\alpha$ is \emph{algebraic over $F$} if $\alpha$ is a root of a polynomial with coefficients in $F$.

  It then follows that $\alpha$ also has a minimal polynomial over $F$
\end{mydefn}

\begin{mydefn}{Algebraic Intersection}
  Suppose $F \subseteq K$ are fields, and $\alpha \in K$ is not \myref{defn:Algebraic} over $F$.

  We set $F(\alpha) := $  the interesction of all intermediate field $F \subseteq E \subseteq K$ such that $\alpha \in E$.
\end{mydefn}

\begin{mydefn}{Splitting Field}
  Given a polynomial $p(x) = (x - \alpha_1) \cdots (x - \alpha_d)$ the field $F(\alpha_1, \ldots, \alpha_d)$ \emph{splitting field for $p(x)$ over $F$}.
\end{mydefn}

\begin{myprop}{Degree Multiplication Theorem}
  Let $F \subseteq K \subseteq L$ be fields.
  Then:
  $[L : F]$ is finite $\iff$ $[L : K]$ and $[K : F]$ are finite,
  and
  $$[L : F] = [L : K] \cdot [K : F]$$
\end{myprop}

\begin{mytheorem}{Primitive Element Theorem}
  Suppose $F \subseteq K$ is an extension of fields satisfying $[K : F] < \infty$.

  If $\Q \subseteq F$.

  Then:
  There exists a primitive element $\alpha \in K$ such that $K = F(\alpha)$
\end{mytheorem}

\begin{myprop}{Splitting Distinct Roots Proposition}
  If $F$ is a field containing $\Q$ and $p(x) \in F[x]$ is irreducible,

  Then:
  $p(x)$ has distinct roots in $K$, the splitting of $p(x)$ over $F$.
\end{myprop}

\begin{mydefn}{Algebraic Extension}
  Let $F \subseteq K$ be an extension of fields.
  \begin{enumerate}[(i)]
    \item $\alpha \in K$ is \emph{algebraic over $F$} if there is a non-constant polynomial $p(x) \in F[x]$ such that $p(\alpha) = 0$.
    \item $K$ is an \emph{algebraic extension of $F$} if every element of $K$ is algebraic over $F$.
  \end{enumerate}
\end{mydefn}

\begin{myprop}{Finite Degree Field Proposition}
  For $F \subseteq K$ fields and $\alpha \in K$, $\alpha$ is \myref{defn:Algebraic} over $K$ $\iff$ $[F(\alpha) : F] < \infty$
\end{myprop}

\begin{mytheorem}{Extended Primitive Element Theorem}
  Suppose $F \subseteq K$ is an extension of fields satisfying $[K : F] < \infty$.
  If $\Q \subseteq F$ or $F$ is finite,
  Then:
  There exists a primitive element $\alpha \in K$ such that $K = F(\alpha)$
\end{mytheorem}

\begin{myprop}{Finite Extension Proposition}
  Let $F \subseteq K$ be a finit extension, with $F$ an infinite field.

  Then:
  there is a primitive element for the extension $\iff$ there are finitely many intermediate fields $F \subseteq E \subseteq K$.
\end{myprop}

\begin{mycorollary}{Finite Intermediate Field Corollary}
  Let $F \subseteq K$ be a finite extension of fields, with $\Q \subseteq F$.

  Then:
  There are only finitely many intermediate fields $F \subseteq E \subseteq K$.
\end{mycorollary}

\begin{myprop}{Algebraic Field Extension}
  Let $F$ be a field.

  Then:
  There exists a field extension $F \subseteq \overline{F}$ with the following property:
  For all $0 \neq f(x) \in F[x]$, there exists $\alpha \in \overline{F}$ such that $f(\alpha) = 0$.
\end{myprop}

\begin{mytheorem}{Algebraic Closure Theorem}
  Let $F$ be a field.

  There is an algebraic extension $F \subseteq \overline{F}$ such that if $p(x) \in F[x]$ and the degree of $p(x)$ is $d > 0$.

  Then:
  There exists $\alpha_1, \ldots, \alpha_d \in F$ (not necessarily distinct) such that $p(x) = (x - \alpha_1) \cdots (x - \alpha_d)$
\end{mytheorem}

\begin{mydefn}{Galois Group}
  The \emph{Galois group} of a field extension $F \subseteq K$:

  It is the set of automorphisms of $K$ fixing $F$.

  If $f(x) \in F[x],\ \alpha \in K$ satisfies $f(\alpha) = 0$, then $f(\sigma(\alpha)) = 0$ for all $\sigma \in Gal(K/F)$.

  If $K = F(\alpha)$ for $\alpha \in K$ is a primitive element, then $Gal(K/F)$ is finite.
  In particular, if $F \subseteq K$ is a finite extension, with $\Q \subseteq F$, then $Gal(K/F)$ is a finite group.
\end{mydefn}

\begin{myprop}{Crucial Proposition}
  Let $F_1 \subseteq K_1, F_2 \subseteq K_2$ be fields.
  $p_1(x) \in F_1[x], p_2(x) \in F_2[x]$ be monic irreducible polynomials of degree $d$, and $\alpha_1 \in K_1, \alpha_2 \in K_2$ roots of $p_1(x), p_2(x)$ (respectively).

  Suppose $\sigma : F_1 \rightarrow F_2$ is an Isomorphism such that $p_2(x) = p_1(x)^{\sigma}$.

  Then:
  There exists an isomorphism $\overline{\sigma} : F_1(\alpha_1) \rightarrow F_2(\alpha_2)$ extending $\sigma$ such that $\overline{\sigma}(\alpha_1) = \alpha_2$
\end{myprop}

Simply stated, this means that the roots of polynomials are rotated by the isomorphisms (and will specifically apply to the automorphisms that are in the \myref{defn:Galois Group}).

We get 2 very nice corollaries of this
\begin{myprop}{Crucial Proposition Corollaries}
  \begin{enumerate}[(i)]
    \item If $p(x) \in F[x]$ is irreducible over $F$ and $\alpha_1, \alpha_2 \in \overline{F}$ are two roots of $p(x)$, then there is an isomorphism from $F(\alpha_1) \rightarrow F(\alpha_2)$ that fixes $F$ and takes $\alpha_1$ to $\alpha_2$
    \item If $K = F(\alpha)$ for $\alpha$ algebraic over $F$, then $|Gal(K/F)|$ equals the number of distinct roots of $p(x)$ in $K$, where $p(x)$ is the minimal polynomial of $\alpha$ over $F$.
  \end{enumerate}
\end{myprop}

\begin{mytheorem}{Galois Theorem}
  Suppose that $F \subseteq K$ is a finit extension with a primitive element, so that $K = F(\alpha)$. Let $p(x)$ denote the minimal polynomial of $\alpha$ over $F$ and write $d = deg(p(x))$.

  Then $K$ is Galois over $F$ $\iff$ $p(x)$ has $d$-distinct roots in $K$
\end{mytheorem}

NOTE: This is important. $K$ is Galois over $F$ if the number of distinct roots is the degree of the minimal polynomial

\begin{mytheorem}{Primitive Galois Theorem}
  Let $K = F(\alpha)$ be a finite extension of $F$ and assume that $K$ is the splitting field of the minimal polynomial of $\alpha$ over $F$.

  Then if $f(x) \in F[x]$ is a non-constant, irreducible polynomial with a root in $K$,
  then $f(x)$ splits over $K$.
\end{mytheorem}

\begin{mydefn}{Fixed Field}
  For $\sigma \in Gal(K/F)$, $K^\sigma := \{ a \in K \mid \sigma(\alpha) = \alpha \}$ is the \emph{fixed field of $\sigma$}.
\end{mydefn}

\begin{mytheorem}{Galois Correspondence Theorem}
  Let $F \subseteq K$ be a finite Galois extension and set $G := Gal(K/F)$.

  Then:
  \begin{enumerate}[(i)]
    \item There is a 1-1 onto, order reversing correspondence between the subgroups $H \subseteq G$ and the intermediate fields $F \subseteq E \subseteq K$ given by $H \rightarrow K^H$ and $E \rightarrow Gal(K/E)$. In particular, for all $H$ and $E$, $H = Gal(K/K^H)$ and $E = K^{Gal(K/E)}$
    \item If $H$ and $E$ correspond, then $[G : H] = [E : F]$
    \item For any intermediate field $E$, $K$ is Galois over $E$.
    \item An intermediate field $E$ is Galois over $F$ $\iff$ $Gal(K/E)$ is a normal subgroup of $G$. In which case, $Gal(E/F) \cong G/Gal(K/E)$
  \end{enumerate}
\end{mytheorem}

\begin{mydefn}{Inverse Galois Problem}
  This is an unsolved problem:

  "Is every finite group the Galois group of a Galois extension of $\Q$?"
\end{mydefn}

\begin{mytheorem}{Finite Galois Extension Theorem}
  Let $G$ be a finite group. Then there exists a finite, Galois extension of field $F \subseteq K$ such that $Gal(K/F) \cong G$.
\end{mytheorem}

% Index
\printindex

\end{document}
